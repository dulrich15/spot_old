

\documentclass{article}

\usepackage[margin=0.625in, rmargin=3in]{geometry}
\usepackage{p200}
\usepackage[colorlinks=true,linkcolor=blue]{hyperref}

\hypersetup{pdftitle = {Physics 203 Homework} }
\hypersetup{pdfauthor = {}, pdfsubject = {Physics} }

\pagestyle{plain}

\begin{document}

\newpage

\begin{center}
\LARGE{Physics 203 Lecture: Electric force and potential} \\[2mm] 
\small{\sf }
\end{center}


\textbf{1.}
% 
\marginpar{\footnotesize\sf \newcounter{listcnt0}
\begin{list}{(\alph{listcnt0})}
{
\usecounter{listcnt0}
\setlength{\rightmargin}{\leftmargin}
}

\item 8.0 yr

\item 6.4 yr

\item 3.6 yr
\end{list}}
%
\quad An earthbound sibling observers her twin to depart from earth at a
speed 60\% the speed of light. After 10.0 years, the travelling twin
returns having travelled 3.0 light-years and back. (a) Calculate the
proper time the travelling twin was aboard the spaceship. (b) From the
traveler's vantage point, the earth-bound twin's clocks tick slow. How
much time does she observe to pass on earth? (c) During the moment of
acceleration 3.0 light-years out, the clocks aboard ship go out of
synch with those on earth. Calculate how far ahead the
desynchronization put the clocks on the ship.
%

\textbf{2.}
% 
\marginpar{\footnotesize\sf \newcounter{listcnt0}
\begin{list}{(\alph{listcnt0})}
{
\usecounter{listcnt0}
\setlength{\rightmargin}{\leftmargin}
}

\item 34.9 AU

\item 74.8 yr
\end{list}}
%
\quad Halley's comet has a highly elongated orbit. The eccentricity of
its orbit is 0.967 with a distance of closest approach equal to
0.586 AU. (a) What is the maximum distance Halley's comet reaches
before it turns around toward the sun? Express your answer in units
of AU. (For reference, Neptune's average distance from the sun is
30 AU.) (b) When distances are measured in astronomical units (AU)
and time is measured in years, Kepler's Third Law simplifies to
$T^2 = a^3$ for orbits around the sun. Determine the period
of Halley's comet in years. (Hint: Halley's original estimate was
76 years).
%

\textbf{3.}
% 
\marginpar{\footnotesize\sf $\pi^-$ = $\bar{u}d$}
%
\quad The pion used to be thought of as the particle that mediates the
attractive interaction between nucleons. For example, a proton can
convert into a neutron by absorbing a negative pion. Knowing that the
quark content of the proton is $uud$ and the neutron is
$udd$, what must the quark content of the negative pion be?
Justify your answer.
%

\textbf{4.}
% 
\marginpar{\footnotesize\sf \newcounter{listcnt0}
\begin{list}{(\alph{listcnt0})}
{
\usecounter{listcnt0}
\setlength{\rightmargin}{\leftmargin}
}

\item $p_x$ = 0 kg-m/s

\item $p_y$ = 0.0684 kg-m/s

\item $KE$ = 10 joules
\end{list}}
%
\quad Figure \hyperref[fig:elas-col]{\ref*{fig:elas-col}} illustrates a possible elastic collision
of two identical particles. They both have an initial speed of
100 m/s, but the final speeds are different. This shows that even
if we could make the molecular speeds in an ideal gas the same,
over time the internal elastic collisions will spread the speed
distribution.


\begin{center}
\def\vx{0}
\def\vy{0.684}
\begin{tikzpicture}
\node (c) at (0,0) [draw,starburst] {};
\node (v1) at ($(c)+(180:2)+(-\vx,-\vy)$) [fill,circle] {};
\node (v2) at ($(c)+(0:2)+(-\vx,-\vy)$) [draw,circle] {};
\node (u1) at ($(c)+(60:2)+(\vx,\vy)$) [fill,circle] {};
\node (u2) at ($(c)+(240:2)+(\vx,\vy)$) [draw,circle] {};
\draw [->-] (v1) -- (c) node [pos=0.5,above=1mm] {$v_1$};
\draw [->-] (v2) -- (c) node [pos=0.5,above=1mm] {$v_2$};
\draw [->-] (c) -- (u1) node [pos=0.5,above left] {$u_1$};
\draw [->-] (c) -- (u2) node [pos=0.5,below right] {$u_2$};
\end{tikzpicture}
\captionof{figure}{Elastic collision}
\label{fig:elas-col}
\end{center}


In this example the initial velocity vectors are
%
\begin{equation*}
v_1 = 100.0 \text{ at }  20\dg \qquad
v_2 = 100.0 \text{ at } 160\dg
\end{equation*}
and the final velocity vectors are
%
\begin{equation*}
u_1 =  66.6 \text{ at } 225\dg \qquad
u_2 = 124.8 \text{ at }  68\dg
\end{equation*}
Confirm that this collision is elastic by showing that both
momentum and kinetic energy are conserved. For your answer,
assume each a mass is one gram.
%

\textbf{5.}
% 
\marginpar{\footnotesize\sf \(5.57 \times 10^{-13}\) tesla}
%
\quad An electron is travelling east at a speed of 100 m/s. It encounters a magnetic field pointing north. What magnitude of magnetic field is required to overcome the acceleration due to gravity effectively levitating the electron?
%


\end{document}