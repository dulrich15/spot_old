
\documentclass{article}

\usepackage[margin=0.625in, rmargin=3in]{geometry}
\usepackage{p200}
\usepackage[colorlinks=true,linkcolor=blue]{hyperref}

\hypersetup{pdftitle = {Physics 203 Homework Solutions} }
\hypersetup{pdfauthor = {}, pdfsubject = {Physics} }

\pagestyle{plain}

\begin{document}

\newpage

\begin{center}
\LARGE{Physics 203 Homework 1} \\[2mm]
\small{\sf Jun 24, 2013}
\end{center}


\textbf{1.}
\marginpar{\footnotesize\sf \newcounter{listcnt0}
\begin{list}{(\alph{listcnt0})}
{
\usecounter{listcnt0}
\setlength{\rightmargin}{\leftmargin}
}

\item 8.0 yr

\item 6.4 yr

\item 3.6 yr
\end{list}}
\quad An earthbound sibling observers her twin to depart from earth at a
speed 60\% the speed of light. After 10.0 years, the travelling twin
returns having travelled 3.0 light-years and back. (a) Calculate the
proper time the travelling twin was aboard the spaceship. (b) From the
traveler's vantage point, the earth-bound twin's clocks tick slow. How
much time does she observe to pass on earth? (c) During the moment of
acceleration 3.0 light-years out, the clocks aboard ship go out of
synch with those on earth. Calculate how far ahead the
desynchronization put the clocks on the ship.
\par \textbf{Solution}
\par (a) The gamma factor for the travelling twin is
%
\begin{equation*}
\gamma = \frac{1}{\sqrt{1 - (0.60)^2}} = 1.25
\end{equation*}
The proper time of the travelling twin is
%
\begin{equation*}
(10) = (1.25) \Delta \tau \implies \tau = 8.0
\end{equation*}
As the earth-bound twin watches, only 8.0 years pass on the ship.

(b) From the \emph{traveler's} point of view it is the earth that is moving
away with a speed of $0.60c$. The gamma factor is also 1.25.
Since the travelling twin is measuring events on earth, her clocks are
not measuring proper time on earth. Using the time dilation formula,
the proper time is
%
\begin{equation*}
(8) = (1.25) \Delta \tau \implies \tau = 6.4
\end{equation*}
In other words, as the travelling twin watches, only 6.4 years pass on
earth.

(c) When the traveller returns either 6.4 years or 10.0 years have
passed. Which is it? The truth is that the trip has two legs with a
brief period of acceleration in the middle. As the question points out,
this acceleration must cover the gap by introducing 3.6 years of
percieved time for the traveller. This is the desynchronization effect.
We can calculate the amount that the clocks on earth appear to shift as:
%
\begin{equation*}
\Delta t = \frac{Lv}{c^2} = \frac{(3.0c)(0.60c)}{c^2} = 1.8
\end{equation*}
Notice the shortcut that $L = 3.0c$. This works because 3.0
light-years is the distance light travels in 3.0 years. Since
$x = vt$, the distance is $L = (c)(3.0)$.

This is the amount of desynchronization introduced by going from rest
to a speed of $0.60c$. But we are slowing down to rest then
accelerating in the other direction up to the same speed so the time
shift is double: 3.6 yr.

\textbf{2.}
\marginpar{\footnotesize\sf \newcounter{listcnt0}
\begin{list}{(\alph{listcnt0})}
{
\usecounter{listcnt0}
\setlength{\rightmargin}{\leftmargin}
}

\item 34.9 AU

\item 74.8 yr
\end{list}}
\quad Halley's comet has a highly elongated orbit. The eccentricity of
its orbit is 0.967 with a distance of closest approach equal to
0.586 AU. (a) What is the maximum distance Halley's comet reaches
before it turns around toward the sun? Express your answer in units
of AU. (For reference, Neptune's average distance from the sun is
30 AU.) (b) When distances are measured in astronomical units (AU)
and time is measured in years, Kepler's Third Law simplifies to
$T^2 = a^3$ for orbits around the sun. Determine the period
of Halley's comet in years. (Hint: Halley's original estimate was
76 years).
\par \textbf{Solution}
\par (a) In terms of orbital parameters we are given $e = 0.967$ and
$r_0 = 0.586$. The equation for the orbit (in polar coordinates)
in general is
%
\begin{equation*}
r = r_0 \frac{1 + e}{1 + e \cos \theta}
\end{equation*}
Clearly when $\theta$ is zero, $r = r_0$ which is the point
of closest approach. When $\theta$ = 180°, we have the point of
farthest distance. Thus,
%
\begin{equation*}
r_1 = r_0 \frac{1 + e}{1 - e} = (0.586)\frac{1.967}{0.033} = 34.93
\end{equation*}
To three significant digits we have $r_1$ = 34.9 AU.

(b) The semi-major axis is the average between closest and farthest
distances. Thus,
%
\begin{equation*}
a = (r_0 + r_1) / 2 = 17.76
\end{equation*}
Using the modified form of Kepler's law we get:
%
\begin{equation*}
T^2 = (17.76)^3 \implies T = 74.8
\end{equation*}

\textbf{3.}
\marginpar{\footnotesize\sf $\pi^-$ = $\bar{u}d$}
\quad The pion used to be thought of as the particle that mediates the
attractive interaction between nucleons. For example, a proton can
convert into a neutron by absorbing a negative pion. Knowing that the
quark content of the proton is $uud$ and the neutron is
$udd$, what must the quark content of the negative pion be?
Justify your answer.
\par \textbf{Solution}
\par In symbols, the reaction we are interested in is:
%
\begin{equation*}
p^+ + \pi^- \rightarrow n
\end{equation*}
The pion is a meson which means that it is a combination of a quark and
an anti-quark. The anti-quark will annihilate one of the quarks in the
proton. The left over quark will combine with the others to form the
neutron. When we write the reaction with quarks, it looks like:
%
\begin{equation*}
uud + \bar{u}d \rightarrow udd
\end{equation*}
The pion must destroy a $u$ quark and bring a $d$.

\textbf{4.}
\marginpar{\footnotesize\sf \newcounter{listcnt0}
\begin{list}{(\alph{listcnt0})}
{
\usecounter{listcnt0}
\setlength{\rightmargin}{\leftmargin}
}

\item $p_x$ = 0 kg-m/s

\item $p_y$ = 0.0684 kg-m/s

\item $KE$ = 10 joules
\end{list}}
\quad Figure \hyperref[fig:elas-col]{\ref*{fig:elas-col}} illustrates a possible elastic collision
of two identical particles. They both have an initial speed of
100 m/s, but the final speeds are different. This shows that even
if we could make the molecular speeds in an ideal gas the same,
over time the internal elastic collisions will spread the speed
distribution.


\begin{center}
\def\vx{0}
\def\vy{0.684}
\begin{tikzpicture}
\node (c) at (0,0) [draw,starburst] {};
\node (v1) at ($(c)+(180:2)+(-\vx,-\vy)$) [fill,circle] {};
\node (v2) at ($(c)+(0:2)+(-\vx,-\vy)$) [draw,circle] {};
\node (u1) at ($(c)+(60:2)+(\vx,\vy)$) [fill,circle] {};
\node (u2) at ($(c)+(240:2)+(\vx,\vy)$) [draw,circle] {};
\draw [->-] (v1) -- (c) node [pos=0.5,above=1mm] {$v_1$};
\draw [->-] (v2) -- (c) node [pos=0.5,above=1mm] {$v_2$};
\draw [->-] (c) -- (u1) node [pos=0.5,above left] {$u_1$};
\draw [->-] (c) -- (u2) node [pos=0.5,below right] {$u_2$};
\end{tikzpicture}
\captionof{figure}{Elastic collision}
\label{fig:elas-col}
\end{center}


In this example the initial velocity vectors are
%
\begin{equation*}
v_1 = 100.0 \text{ at }  20\dg \qquad
v_2 = 100.0 \text{ at } 160\dg
\end{equation*}
and the final velocity vectors are
%
\begin{equation*}
u_1 =  66.6 \text{ at } 225\dg \qquad
u_2 = 124.8 \text{ at }  68\dg
\end{equation*}
Confirm that this collision is elastic by showing that both
momentum and kinetic energy are conserved. For your answer,
assume each a mass is one gram.
\par \textbf{Solution}
\par We will need to calculate some components. Let's just do them
all at once right now using $v_x = v \cos \theta$ and
$v_y = v \sin \theta$:


\begin{center}

\renewcommand{\arraystretch}{1.5}
\renewcommand{\tabcolsep}{0.2cm}

\begin{tabular}{ccccc}
\hline
\textbf{Vector} & \textbf{Mag.} & \textbf{Ang.} & \textbf{$x$} & \textbf{$y$} \\ 
\hline
 $v_1$ &  100.0 &  20° &  93.97 &  34.20 \\ 
 $v_2$ &  100.0 &  160° &  -93.97 &  34.20 \\ 
 $u_1$ &  66.6 &  225° &  -46.98 &  -47.18 \\ 
 $u_2$ &  124.8 &  68° &  46.98 &  115.58 \\ 
\hline
\end{tabular}



\end{center}


It is now pretty straight-forward to calculate the momentum and
kinetic energy for each particle:


\begin{center}

\renewcommand{\arraystretch}{1.5}
\renewcommand{\tabcolsep}{0.2cm}

\begin{tabular}{cccc}
\hline
\textbf{Vector} & \textbf{$p_x$} & \textbf{$p_y$} & \textbf{$KE$} \\ 
\hline
 $v_1$ &  0.09397 &  0.03420 &  5.0000 \\ 
 $v_2$ &  -0.09397 &  0.03420 &  5.0000 \\ 
 $u_1$ &  -0.04698 &  -0.04718 &  2.2166 \\ 
 $u_2$ &  0.04698 &  0.11558 &  7.7834 \\ 
\hline
\end{tabular}



\end{center}


Adding up the first pair of rows give the initial quantities:
%
\begin{equation*}
p_x = 0 \qquad \text{and} \qquad p_y = 0.06840
\end{equation*}
and
%
\begin{equation*}
KE = 10.0000
\end{equation*}
Adding up the second pair of rows gives the same numbers which
shows that both momentum and kinetic energy are conserved in
this collision.

\textbf{5.}
\marginpar{\footnotesize\sf \(5.57 \times 10^{-13}\) tesla}
\quad An electron is travelling east at a speed of 100 m/s. It encounters a magnetic field pointing north. What magnitude of magnetic field is required to overcome the acceleration due to gravity effectively levitating the electron?
\par \textbf{Solution}
\par The force experienced due to gravity is simply the weight:
%
\begin{equation*}
W = mg = (\sci{9.109}{-31})(9.8) = \sci{8.927}{-30}
\end{equation*}
The magnetic force needs to counter-balance this force. The magnitude of the magnetic force is given by $F = qvB$. In this case:
%
\begin{equation*}
(\sci{8.927}{-30}) = (\sci{1.602}{-19})(100)(B) \implies B = \sci{5.57}{-13}
\end{equation*}
This is a miniscule amount of field. This shows how much more powerful the electromagnetic force is than gravity


\end{document}